\documentclass[twocolumn]{article}
\usepackage[utf8]{inputenc}
\usepackage[T1]{fontenc}
\usepackage[spanish]{babel}
\usepackage{amsmath}
\usepackage{graphicx}
\usepackage{hyperref}
% \usepackage{natbib}

\title{Modelo de Ising con el Algoritmo de Metrópolis}
\author{Tu Nombre}
\date{}

\begin{document}

\maketitle

\begin{abstract}
\end{abstract}

\section{Introducción}

\section{Marco Teórico}
\subsection*{El Modelo de Ising}

El Modelo de Ising es un modelo matemático de ferromagnetismo en física estadística. Propuesto por Wilhelm Lenz en 1920 y nombrado después de su estudiante Ernst Ising, este modelo es uno de los más simples que muestra una transición de fase \cite{isingwiki}.

En su forma más básica, el modelo consiste en spines discretos que pueden estar en uno de dos estados (+1 o -1) y que están dispuestos en una red, generalmente una red cuadrada en dos dimensiones. Cada spin interactúa con sus vecinos más cercanos. La energía de un estado particular del sistema está dada por:

\begin{equation}
    E = -J \sum_{\langle i,j \rangle} s_i s_j - H \sum_i s_i
\end{equation}

donde $s_i$ es el valor del spin en el sitio $i$, $J$ es la constante de acoplamiento entre spines vecinos, $H$ es el campo magnético externo, y la primera suma se realiza sobre todos los pares de spines vecinos más cercanos.

El Modelo de Ising es particularmente importante porque, a pesar de su simplicidad, exhibe un comportamiento de transición de fase, mostrando ferromagnetismo por debajo de una temperatura crítica en dos o más dimensiones.

\subsection*{Cálculo del Calor Específico}

El calor específico por spin, C, se calcula utilizando las fluctuaciones en la energía del sistema. La fórmula para el calor específico es \cite{chang_fisica_computacional}:

\begin{equation}
    C = \frac{\beta}{T} (\langle E^2 \rangle - \langle E \rangle^2)
\end{equation}

donde $\beta = 1/(k_B T)$ es la temperatura inversa, $\langle E \rangle$ es el promedio de la energía, y $\langle E^2 \rangle$ es el promedio del cuadrado de la energía. 

\subsection*{Cálculo de la Susceptibilidad Magnética}

De manera similar, la susceptibilidad magnética $\chi$ se calcula utilizando las fluctuaciones en la magnetización del sistema. La fórmula para la susceptibilidad magnética es \cite{chang_fisica_computacional}:

\begin{equation}
    \chi = \beta (\langle M^2 \rangle - \langle M \rangle^2)
\end{equation}

donde $\langle M \rangle$ es el promedio de la magnetización, y $\langle M^2 \rangle$ es el promedio del cuadrado de la magnetización. La magnetización $M$ se define como la suma de todos los spines en el sistema: $M = \sum_i s_i$.

\subsection*{El Algoritmo de Metrópolis}

El algoritmo de Metrópolis es una técnica de muestreo de Monte Carlo ampliamente utilizada en física estadística para simular sistemas complejos como el Modelo de Ising \cite{algorithmarchive}. Este método permite explorar eficientemente el espacio de configuraciones del sistema, especialmente en situaciones donde el cálculo directo de las propiedades termodinámicas es computacionalmente prohibitivo.

El algoritmo funciona de la siguiente manera:

\begin{enumerate}
    \item Se inicia con una configuración aleatoria del sistema.
    \item Se propone un cambio aleatorio en la configuración (por ejemplo, voltear un spin).
    \item Se calcula el cambio de energía $\Delta E$ asociado con este cambio.
    \item Si $\Delta E \leq 0$, se acepta el cambio.
    \item Si $\Delta E > 0$, se acepta el cambio con una probabilidad $e^{-\beta \Delta E}$, donde $\beta = 1/(k_B T)$, $k_B$ es la constante de Boltzmann y $T$ es la temperatura.
    \item Se repiten los pasos 2-5 muchas veces para alcanzar el equilibrio térmico.
\end{enumerate}

Este proceso permite al sistema evolucionar hacia el equilibrio térmico, muestreando configuraciones de acuerdo con la distribución de Boltzmann. Es particularmente eficaz para el Modelo de Ising, ya que permite simular el comportamiento del sistema a diferentes temperaturas y estudiar fenómenos como la transición de fase ferromagnética.


\section{Metodología}

Para simular el Modelo de Ising en dos dimensiones, emplearemos una representación matricial implementada con la biblioteca NumPy de Python. Nuestro enfoque se basa en la creación de una red cuadrada de longitud L, representada mediante un arreglo NumPy de forma L×L. Esta estructura bidimensional nos permite modelar de manera eficiente la disposición espacial de los spines en el sistema.

En esta representación, cada elemento del arreglo corresponde a un spin individual. Los valores de estos elementos se limitan a +1 o -1, simbolizando spines orientados hacia arriba o hacia abajo, respectivamente. Esta dicotomía refleja la naturaleza binaria del Modelo de Ising, capturando la esencia de los estados magnéticos en un material ferromagnético.

La elección de esta estructura matricial no es arbitraria. La forma del arreglo nos brinda una ventaja significativa al facilitar el acceso a los vecinos más cercanos de cada spin. Esta característica es fundamental para nuestro estudio, ya que las interacciones entre spines adyacentes son el núcleo del comportamiento del sistema Ising. La capacidad de calcular rápidamente estas interacciones locales es crucial para implementar eficientemente el algoritmo de Metrópolis y simular con precisión la evolución del sistema.

Esta representación matricial nos proporciona una estructura de datos no solo eficiente, sino también intuitiva y fácil de manipular. Nos permite implementar el algoritmo de Metrópolis de manera directa, facilitando la simulación de la dinámica del sistema Ising a lo largo del tiempo y bajo diferentes condiciones. Con esta base, podemos proceder a explorar cómo el sistema responde a variaciones en la temperatura, el campo magnético externo, y otros parámetros relevantes del modelo.

Esta representación nos proporciona una estructura de datos eficiente y fácil de manipular para implementar el algoritmo de Metrópolis y simular la evolución del sistema Ising.



\section{Resultados}

\section{Discusión}

\section{Conclusiones}

\bibliographystyle{plain}
\bibliography{referencias}

\end{document}